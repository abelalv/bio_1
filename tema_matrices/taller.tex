\documentclass{article}
\usepackage{amsmath}

\title{Taller de Álgebra Lineal}
\date{}

\begin{document}

\maketitle

\section*{Problemas}

\begin{enumerate}
\item  Sea \( A \in \mathbb{R}^{2 \times 2} \) y \( B \in \mathbb{R}^{2 \times 2} \), tales que:
\[
A = \begin{pmatrix} 1 & 2 \\ 3 & 4 \end{pmatrix}, \quad B = \begin{pmatrix} 5 & 6 \\ 7 & 8 \end{pmatrix}.
\]
Calcule la suma \( A + B \).

\item Dadas las matrices \( C \in \mathbb{R}^{3 \times 3} \) y \( D \in \mathbb{R}^{3 \times 3} \):
\[
C = \begin{pmatrix} 1 & 0 & 2 \\ 3 & -1 & 4 \\ 0 & 5 & 6 \end{pmatrix}, \quad D = \begin{pmatrix} 2 & 3 & 1 \\ -2 & 0 & 4 \\ 1 & 7 & 3 \end{pmatrix}.
\]
Determine \( C + D \).

\item Sea la matriz \( E \in \mathbb{R}^{2 \times 2} \):
\[
E = \begin{pmatrix} 4 & -2 \\ 1 & 3 \end{pmatrix}.
\]
Calcule \( 3E \).

\item  Dada la matriz \( F \in \mathbb{R}^{3 \times 3} \):
\[
F = \begin{pmatrix} 2 & 1 & 0 \\ 4 & -1 & 2 \\ 3 & 5 & -3 \end{pmatrix},
\]
calcule \( -2F \).

\item Sea \( G \in \mathbb{R}^{2 \times 3} \):
\[
G = \begin{pmatrix} 1 & 4 & 6 \\ 3 & 5 & 2 \end{pmatrix}.
\]
Encuentre \( G^T \).



\item Dada la matriz \( H \in \mathbb{R}^{3 \times 2} \):
\[
H = \begin{pmatrix} 2 & 5 \\ 0 & 1 \\ 4 & 3 \end{pmatrix},
\]
calcule \( H^T \).



\item 
Sean las matrices \( I \in \mathbb{R}^{2 \times 2} \) y \( J \in \mathbb{R}^{2 \times 2} \):
\[
I = \begin{pmatrix} 1 & 3 \\ 2 & 4 \end{pmatrix}, \quad J = \begin{pmatrix} 2 & 0 \\ 1 & -1 \end{pmatrix}.
\]
Calcule \( 2I + 3J \).


\item Dadas las matrices \( K \in \mathbb{R}^{3 \times 3} \) y \( L \in \mathbb{R}^{3 \times 3} \):
\[
K = \begin{pmatrix} 1 & 0 & 2 \\ 4 & 1 & 3 \\ 0 & 5 & -1 \end{pmatrix}, \quad L = \begin{pmatrix} 2 & 3 & 1 \\ 0 & 4 & 2 \\ 1 & -1 & 5 \end{pmatrix},
\]
determine \( 4K - L \).


\item Sea \( M \in \mathbb{R}^{2 \times 3} \) y \( N \in \mathbb{R}^{2 \times 3} \):
\[
M = \begin{pmatrix} 1 & 0 & 3 \\ 4 & 5 & 6 \end{pmatrix}, \quad N = \begin{pmatrix} 2 & 3 & 1 \\ 0 & -2 & 4 \end{pmatrix}.
\]
Calcule \( M^T + N^T \).



 \item Sea \( O \in \mathbb{R}^{2 \times 3} \):
\[
O = \begin{pmatrix} 2 & 3 & 4 \\ 1 & 5 & 6 \end{pmatrix}.
\]
Calcule \( 2O^T \).

\end{enumerate}

\end{document}
