% Options for packages loaded elsewhere
\PassOptionsToPackage{unicode}{hyperref}
\PassOptionsToPackage{hyphens}{url}
\PassOptionsToPackage{dvipsnames,svgnames,x11names}{xcolor}
%
\documentclass[
]{article}

\usepackage{amsmath,amssymb}
\usepackage{iftex}
\ifPDFTeX
  \usepackage[T1]{fontenc}
  \usepackage[utf8]{inputenc}
  \usepackage{textcomp} % provide euro and other symbols
\else % if luatex or xetex
  \usepackage{unicode-math}
  \defaultfontfeatures{Scale=MatchLowercase}
  \defaultfontfeatures[\rmfamily]{Ligatures=TeX,Scale=1}
\fi
\usepackage{lmodern}
\ifPDFTeX\else  
    % xetex/luatex font selection
\fi
% Use upquote if available, for straight quotes in verbatim environments
\IfFileExists{upquote.sty}{\usepackage{upquote}}{}
\IfFileExists{microtype.sty}{% use microtype if available
  \usepackage[]{microtype}
  \UseMicrotypeSet[protrusion]{basicmath} % disable protrusion for tt fonts
}{}
\makeatletter
\@ifundefined{KOMAClassName}{% if non-KOMA class
  \IfFileExists{parskip.sty}{%
    \usepackage{parskip}
  }{% else
    \setlength{\parindent}{0pt}
    \setlength{\parskip}{6pt plus 2pt minus 1pt}}
}{% if KOMA class
  \KOMAoptions{parskip=half}}
\makeatother
\usepackage{xcolor}
\usepackage[margin=1in]{geometry}
\setlength{\emergencystretch}{3em} % prevent overfull lines
\setcounter{secnumdepth}{5}
% Make \paragraph and \subparagraph free-standing
\makeatletter
\ifx\paragraph\undefined\else
  \let\oldparagraph\paragraph
  \renewcommand{\paragraph}{
    \@ifstar
      \xxxParagraphStar
      \xxxParagraphNoStar
  }
  \newcommand{\xxxParagraphStar}[1]{\oldparagraph*{#1}\mbox{}}
  \newcommand{\xxxParagraphNoStar}[1]{\oldparagraph{#1}\mbox{}}
\fi
\ifx\subparagraph\undefined\else
  \let\oldsubparagraph\subparagraph
  \renewcommand{\subparagraph}{
    \@ifstar
      \xxxSubParagraphStar
      \xxxSubParagraphNoStar
  }
  \newcommand{\xxxSubParagraphStar}[1]{\oldsubparagraph*{#1}\mbox{}}
  \newcommand{\xxxSubParagraphNoStar}[1]{\oldsubparagraph{#1}\mbox{}}
\fi
\makeatother


\providecommand{\tightlist}{%
  \setlength{\itemsep}{0pt}\setlength{\parskip}{0pt}}\usepackage{longtable,booktabs,array}
\usepackage{calc} % for calculating minipage widths
% Correct order of tables after \paragraph or \subparagraph
\usepackage{etoolbox}
\makeatletter
\patchcmd\longtable{\par}{\if@noskipsec\mbox{}\fi\par}{}{}
\makeatother
% Allow footnotes in longtable head/foot
\IfFileExists{footnotehyper.sty}{\usepackage{footnotehyper}}{\usepackage{footnote}}
\makesavenoteenv{longtable}
\usepackage{graphicx}
\makeatletter
\def\maxwidth{\ifdim\Gin@nat@width>\linewidth\linewidth\else\Gin@nat@width\fi}
\def\maxheight{\ifdim\Gin@nat@height>\textheight\textheight\else\Gin@nat@height\fi}
\makeatother
% Scale images if necessary, so that they will not overflow the page
% margins by default, and it is still possible to overwrite the defaults
% using explicit options in \includegraphics[width, height, ...]{}
\setkeys{Gin}{width=\maxwidth,height=\maxheight,keepaspectratio}
% Set default figure placement to htbp
\makeatletter
\def\fps@figure{htbp}
\makeatother

\makeatletter
\@ifpackageloaded{caption}{}{\usepackage{caption}}
\AtBeginDocument{%
\ifdefined\contentsname
  \renewcommand*\contentsname{Table of contents}
\else
  \newcommand\contentsname{Table of contents}
\fi
\ifdefined\listfigurename
  \renewcommand*\listfigurename{List of Figures}
\else
  \newcommand\listfigurename{List of Figures}
\fi
\ifdefined\listtablename
  \renewcommand*\listtablename{List of Tables}
\else
  \newcommand\listtablename{List of Tables}
\fi
\ifdefined\figurename
  \renewcommand*\figurename{Figure}
\else
  \newcommand\figurename{Figure}
\fi
\ifdefined\tablename
  \renewcommand*\tablename{Table}
\else
  \newcommand\tablename{Table}
\fi
}
\@ifpackageloaded{float}{}{\usepackage{float}}
\floatstyle{ruled}
\@ifundefined{c@chapter}{\newfloat{codelisting}{h}{lop}}{\newfloat{codelisting}{h}{lop}[chapter]}
\floatname{codelisting}{Listing}
\newcommand*\listoflistings{\listof{codelisting}{List of Listings}}
\makeatother
\makeatletter
\makeatother
\makeatletter
\@ifpackageloaded{caption}{}{\usepackage{caption}}
\@ifpackageloaded{subcaption}{}{\usepackage{subcaption}}
\makeatother

\ifLuaTeX
  \usepackage{selnolig}  % disable illegal ligatures
\fi
\usepackage{bookmark}

\IfFileExists{xurl.sty}{\usepackage{xurl}}{} % add URL line breaks if available
\urlstyle{same} % disable monospaced font for URLs
\hypersetup{
  pdftitle={Pontificia Universidad Javeriana},
  pdfauthor={Matemáticas Para Biologia I},
  colorlinks=true,
  linkcolor={blue},
  filecolor={Maroon},
  citecolor={Blue},
  urlcolor={Blue},
  pdfcreator={LaTeX via pandoc}}


\title{Pontificia Universidad Javeriana}
\author{Matemáticas Para Biologia I}
\date{2024-09-30}

\begin{document}
\maketitle


\textbf{Taller derivada implicita}

\begin{enumerate}
\def\labelenumi{\arabic{enumi}.}
\item
  Halla la derivada de \(y\) con respecto a \(x\) de la siguiente
  ecuación implícita: \[ x^2 + y^2 = 25 \]
\item
  Dada la ecuación \(x^3 + y^3 = 6xy\), encuentra \(\frac{dy}{dx}\).
\item
  En biología, el crecimiento de dos poblaciones está relacionado de la
  siguiente manera: \[ x^2 + xy + y^2 = 50 \] donde \(x\) representa la
  población de una especie y \(y\) representa la población de otra
  especie. Encuentra \(\frac{dy}{dx}\) y explica qué significa en
  términos del cambio relativo entre ambas poblaciones.
\item
  Para la siguiente curva implícita, encuentra \(\frac{dy}{dx}\):
  \[ e^{x} + y^2 = 2xy \]
\item
  Supongamos que la concentración de una sustancia en una célula está
  dada implícitamente por la ecuación \(x^2y + xy^2 = 10\), donde \(x\)
  representa el tiempo y \(y\) la concentración. Encuentra la tasa de
  cambio de la concentración con respecto al tiempo \(\frac{dy}{dx}\) en
  términos de \(x\) y \(y\).
\item
  Dada la ecuación implícita \(2x + 3\sin(y) = x^2y\), encuentra
  \(\frac{dy}{dx}\).
\item
  Resuelve la derivada implícita de la ecuación \(xy + \ln(y) = 3\).
\item
  En un modelo biológico, la relación entre dos variables \(x\)
  (proporción de presas) y \(y\) (proporción de depredadores) está dada
  por la ecuación implícita: \[ x + y^3 = xy \] Encuentra
  \(\frac{dy}{dx}\) y discute cómo un cambio en la proporción de presas
  afecta la proporción de depredadores.
\item
  \textbf{Crecimiento de una población} La población de una colonia de
  bacterias se duplica cada 3 horas. Si inicialmente hay 500 bacterias,
  la población \(P(t)\) en cualquier tiempo \(t\) (en horas) está dada
  por la fórmula de crecimiento exponencial: \[ P(t) = P_0 e^{kt} \]
  donde \(P_0\) es la población inicial y \(k\) es la constante de
  crecimiento.
\end{enumerate}

\begin{itemize}
\item
  Encuentra el valor de \(k\).
\item
  ¿Cuál será la población después de 9 horas?
\item
  ¿Cuánto tiempo tomará para que la población alcance las 8000
  bacterias?
\end{itemize}

\begin{enumerate}
\def\labelenumi{\arabic{enumi}.}
\setcounter{enumi}{10}
\tightlist
\item
  \textbf{Descomposición radiactiva} Una sustancia radiactiva se
  descompone siguiendo una ley de descomposición exponencial. Si la
  cantidad inicial de la sustancia es 120 gramos y se sabe que después
  de 5 horas solo quedan 75 gramos, la cantidad de sustancia restante en
  cualquier momento \(t\) está dada por: \[ N(t) = N_0 e^{-kt} \]
\end{enumerate}

\begin{itemize}
\item
  Encuentra el valor de la constante de descomposición \(k\).
\item
  ¿Cuánta sustancia quedará después de 10 horas?
\item
  ¿En cuánto tiempo se descompondrá el 90\% de la sustancia?
\end{itemize}




\end{document}
